

%=skak version 1.5
\chapter*{Who Strikes First?}



%\makeatletter
%\def\afterno{\hspace{1ex}}
%\def\beforeno{\hspace{6ex}}
%\def\beforeb{\beforeno\the\move\afterno\dots.\hspace{5ex}}

%\renewcommand*\ECO[1]{\def\@eco{\ (#1)}}
%\renewcommand*\whiteelo[1]{\def\@welo{\ (#1)}}
%\renewcommand*\blackelo[1]{\def\@belo{\ (#1)}}
%\renewcommand*\makegametitle{\newgame
%\begin{center}
%\noindent\textbf{\wname}\@welo\ -- \textbf{\bname}\@belo\\
%\noindent\textit{\@tourn}\\
%\noindent\@opening\@eco
%\end{center}}
%\makeatother

%\whitename{Kupper}\whiteelo{1234}
%\blackname{Tal}\blackelo{4321}
%\chessevent{Zurich - 1959}
%\ECO{B96n}
%\chessopening{Sicilian Najdorf, Polugaevsky Var}




%\makegametitle

\newchessgame[id=main,white=Kupper,black=Tal,result=0--1,event=Zurich,date=1959]

\mainline[style=styleC,level=1]{1. e4}
%
The struggle for the center is not always conducted with a
symmetrical Pawn formation. There are other popular answers to
\xskakset{moveid=1w}%
$\xskakget{opennr}$%
\xskakget{san}.
One of them is the Sicilian Defense.
%
\mainline{1... c5}
%
The characteristic Sicilian move.
%
\mainline{2. Nf3}
%
One usually develops the g\figsymbol{N} when his e\figsymbol{p} is not under attack.
%
\mainline{2... d6}
%
\hidemoves{3. d4}
%
%g\figsymbol{N}.
%
Now after developing the White-squared Bishop and castling,
White could decide where to put his Black-squared Bishop, but in
practice he prefers to open the center first. Of course, a good
preparatory move is %
\newchessgame[newvar=main,id=variation]%
\mainline[level=2]{3. c3,} so as after \wmove{d4}
to recapture with a \figsymbol{p}. But after \mainline{3... Nf6}
defending the e\figsymbol{p} is awkward (\mainline{4. Qc2? e5!}).

\resumechessgame[id=main,moveid=3w]

\mainline[level=1]{3. d4} 

Now Black must exchange his c\figsymbol{p}.

\mainline{3... cxd4}

The \figsymbol{Q} does not recapture the \figsymbol{p} because she would be
attacked by enemy pieces brought into play with tempo.

\mainline{4. Nxd4}

Therefore, the \figsymbol{N} takes the Pawn.

\mainline{4... Nf6}

To attack the enemy e\figsymbol{p}.

\mainline{5. Nc3}

%\hidemoves{5... a6}

Where is Black to place his black-squared Bishop? He could
fianchettoe it 
(\mainline[level=2]{5... g6}, \hidemoves{6. g3}\mainline{6... Bg7}, \hidemoves{7. Bg2} \mainline{7... O-O} -- the Dragon Variation)
or
after \bmove{e6} (The Scheveningen Variation) or \bmove{e5}
(The Boleslavsky System) place it on e7.
All these are fully possible, but Black first decides to keep White pieces from b5
by moving a Queenside Pawn.

\resumechessgame[id=main,moveid=5b]
\mainline[level=1]{5... a6}

\hidemoves{6. Bg5}

\xskakset{level=2}
White has a choice of developing continuations
(\variation{6.Be2}, \variation{6.Bc4}, \variation{6.f4}, \variation{6.b3})
but chooses an idea involved with Queenside castling.

\resumechessgame[id=main,moveid=6w]

\mainline[level=1]{6. Bg5}
%
Continuing his plan.
%
\mainline{6... e6}
%
Often played here is \hidemoves{7. Qd2}\variation[invar]{7. Qd2}, but this can be met by
the tactical \hidemoves{7... h6}\variation{7... h6 8. Bh4 Nxe4!}
\resumechessgame[id=main,moveid=7w]%
Therefore, \hidemoves{7. Qf3}\variation{7. Qf3}
comes into consideration (to enable Queenside castling).
But first White stops to place another Pawn in the center.
%
\resumechessgame[id=main,moveid=7w]
\mainline[level=1]{7. f4}

\chessboard[smallboard]

\hidemoves{7... b5}
Is there a threat of \variation[invar]{8. e5}? No, this is not true,
as after \hidemoves{8. e5}\variation{8... dxe5 9. fxe5}\hidemoves{8... dxe5 9. fxe5} may be met by
\variation{9... Qa5} and \variation{9... Qc7}
and even \variation{9... h6}.  This continuation is more dangerous 
after \resumechessgame[id=main,moveid=7b]\variation{7... Nc6 8. Nxc6 bxc6}, but \variation{7... Be7} meets the threat.
The move \variation{7... Qb6}\hidemoves{7... Qb6} is involved with a Pawn sacrifice variation 
(\variation{8. Qd2 Qxb2 9. Rb1 Qa3 \xskakcomment{ and now} 10. e5 Nfd7 11. f5}).  
Thus Black does not need to defend e5 by \bmove{Nc6}.
%
\resumechessgame[id=main,moveid=7b]
\mainline[level=1]{ 7... b5} 
%
Black continues aggressively.
%
\mainline{8. Qf3}

\xskakset{level=2}
White defends his e\figsymbol{p} against a threat of \bmove{b4}.

\mainline[level=1]{8... Bb7}

\xskakset{level=2}
Hoping to induce \wmove{a3} but after Queenside castling, the \figsymbol{p} on \wmove{a3}
facilitates Black opening play on
the Queenside by \bmove{b4}.  So White protects his e\figsymbol{p} with a piece.

\mainline[level=1]{9. Bd3}

Best.

\mainline{9...Be7}

Continuing his development.

\mainline{10. O-O-O}

Continuing his plan.

\mainline{10... Qb6}

\mainline{11. Rhe1}

Developing and centralizing the White \movecomment{R.}

\mainline{11... Nbd7}

\mainline{12. Nce2}

\mainline{12... Nc5}

\mainline{13. Bxf6}

\mainline{13... Bxf6}

\mainline{14. g4}

\showboard

\mainline{14... Na4}

\mainline{15. c3}

\mainline{15... b4}

\mainline{16. Bc2}

\mainline{16... Nxb2} 

\mainline{17. Kxb2}

\mainline{17... bxc3+}

\mainline{18. Kxc3}

\showboard

\mainline{18... O-O}

\mainline{19. Rb1}

\mainline{19... Qa5+}

\mainline{20. Kd3}

\mainline{20... Rac8}

\mainline{21. Qf2}

\mainline{21... Ba8}

\chessboard

\mainline{22.	Rb3}

\mainline{22... e5}

\mainline{23.	g5}

\mainline{23... exd4}

\mainline{24.	Nxd4}

\mainline{24... Bxd4}

\showboard
